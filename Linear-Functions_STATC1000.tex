\documentclass{article}   
\usepackage{tikz}
\usepackage{amsmath}
\usepackage[margin=1in, top=0.8in]{geometry}
\usepackage{pgfplots} 
\usepackage{pgfplotstable}
\usepackage{titling}

\pgfplotsset{compat=1.18} 


\setlength{\droptitle}{-6em}               
\title{Linear Regressions}  
\author{Eric Jones}      
\date{September 3, 2025}      

\begin{document}    

\maketitle                   

%%%%%%%%%% Page 1 %%%%%%%%%%
\section{Review: Slope-intercept Form}      
Slope-intercept form is the equation of a straight line written as \(y = mx + b\) where \(x\) represents the line's slope and \(b\) represents the y-intercept.

\subsection{Examples of Slope-Intercept and Solving for Y}
Example 1 (the given equation is already in the correct format):
\[y = 3x + 7\]
\vspace{-1.5em}
\[\text{slope = $3$}\]
\vspace{-1.5em}
\[\text{y-intercept = $7$}\]
\vspace{1em}
\noindent Example 2 (we need to solve for $y$):
\[
\begin{aligned}
3x + 4y &= 5 \\
4y &= 5 - 3x \\
y &= \frac{5 - 3x}{4} \\
y &= \frac{-3x}{4} + \frac{5}{4}
\end{aligned}
\]
\vspace{-0.5em}
\vspace{0.5em}
\[\text{slope = $\frac{-3}{4}$}\]
\vspace{-1em}
\[\text{y-intercept = $\frac{5}{4}$}\]
\vfill
\begin{center}
       (Continues on next page.)
\end{center}

%%%%%%%%%% Page 2 %%%%%%%%%%
\newpage
\section{Paired Data}
A data set where each inidividual is described by two variables.

\noindent You can visualize paired data with a scatter chart.

\begin{center}
\begin{tikzpicture}
\begin{axis}[
    xlabel={$x$},
    ylabel={$y$},
    title={Scatter Plot},
    grid=both,
    width=8cm,
    height=6cm,
    xtick=\empty,
    ytick=\empty,
]
% Scatter Points
\addplot[
    only marks,
    mark=*,
] table[
    col sep=comma,
    x=X,
    y=Y
] {basic_scatter_plot.csv};
\end{axis}
\end{tikzpicture}
\end{center}

Scatter plots help us:
\begin{itemize}
       \item Identify if the data has a trend or pattern
       \item Identify if two variables are \emph{correlated} or not
       \item Make predictions
\end{itemize}

\subsection{Strong Correlation Example}
%% Height vs Shoe Size Scatter Plot
% Graph Outline
\begin{center}
\begin{tikzpicture}
\begin{axis} [
       xlabel={Shoe Size},
       ylabel={Height},
       title={Height vs. Shoe Size},
       width=12cm,
       height=8cm,
]
% Scatter Points
\addplot[
    only marks,
    mark=*,
] table[
    col sep=comma,
    x=ShoeSize,
    y=Height
] {shoesize_height.csv};
% Regression Line
\addplot[
    red,
    thick,
    opacity=0.4
] table[
    col sep=comma,
    x=ShoeSize,
    y=Height,
    y={create col/linear regression={y=Height}}
] {shoesize_height.csv};
\end{axis}
\end{tikzpicture}
\end{center}

When we analyze this scatter plot, we can see:
\begin{itemize}
       \item There is a \emph{strong correlation} between height and shoe size
       \item This allows us to make a prediction of somebody's height based on their shoe size
\end{itemize}
\vfill
\begin{center}
       (Continues on next page.)
\end{center}

%%%%%%%%%% Page 3 %%%%%%%%%%
\newpage
\subsection{Weak or No Correlation Example}
%% Age vs Height Scatter Plot
% Graph Outline
\begin{center}
\begin{tikzpicture}
\begin{axis} [
       xlabel={Age},
       ylabel={Height},
       title={Height vs. Age},
       width=12cm,
       height=8cm,
]
% Scatter Points
\addplot[
    only marks,
    mark=*,
] table[
    col sep=comma,
    x=Age,
    y=Height
] {age_height.csv};
% Regression Line
\addplot[
    red,
    thick,
    opacity=0.4
] table[
    col sep=comma,
    x=Age,
    y=Height,
    y={create col/linear regression={y=Height}}
] {age_height.csv};
\end{axis}
\end{tikzpicture}
\end{center}

When we analyze this scatter plot, we can see:
\begin{itemize}
       \item There is either a \emph{weak correlation} or \emph{no correlation }between height and age
       \item The age variable does not seem to influence the height variable
\end{itemize}

\section{Types of Relationships in Paired Data}
\subsection{Positive Linear Correlation}
%% Positive Linear Correlation - Scatter Plot
% Graph Outline
\begin{center}
\begin{tikzpicture}
\begin{axis}[
    xlabel={$x$},
    ylabel={$y$},
    title={Positive Linear Correlation},
    grid=both,
    width=8cm,
    height=6cm,
    xtick=\empty,
    ytick=\empty,
]
% Scatter Points
\addplot[
    only marks,
    mark=*,
] table[
    col sep=comma,
    x=X,
    y=Y
] {positive_linear_correlation.csv};
% Regression Line
\addplot[
    red,
    thick,
    opacity=0.4
] table[
    col sep=comma,
    x=X,
    y=Y,
    y={create col/linear regression={y=Y}}
] {positive_linear_correlation.csv};
\end{axis}
\end{tikzpicture}
\end{center}

When we analyze this scatter plot, we can see:
\begin{itemize}
       \item The data has a clear linear shape
       \item The slope of the line is \emph{positive}
\end{itemize}
\vfill
\begin{center}
       (Continues on next page.)
\end{center}

%%%%%%%%%% Page 4 %%%%%%%%%%
\newpage
\subsection{Negative Linear Correlation}
%% Negative Linear Correlation - Scatter Plot
% Graph Outline
\begin{center}
\begin{tikzpicture}
\begin{axis}[
    xlabel={$x$},
    ylabel={$y$},
    title={Negative Linear Correlation},
    grid=both,
    width=8cm,
    height=6cm,
    xtick=\empty,
    ytick=\empty,
]
% Scatter Points
\addplot[
    only marks,
    mark=*,
] table[
    col sep=comma,
    x=X,
    y=Y
] {negative_linear_correlation.csv};
% Regression Line
\addplot[
    red,
    thick,
    opacity=0.4
] table[
    col sep=comma,
    x=X,
    y=Y,
    y={create col/linear regression={y=Y}}
] {negative_linear_correlation.csv};
\end{axis}
\end{tikzpicture}
\end{center}

When we analyze this scatter plot, we can see:
\begin{itemize}
       \item The data has a clear linear shape
       \item The slope of the line is \emph{negative}
\end{itemize}

\subsection{Non Linear Correlation}
%% Non Linear Correlation - Scatter Plot
% Graph Outline
\begin{center}
\begin{tikzpicture}
\begin{axis}[
    xlabel={$x$},
    ylabel={$y$},
    title={Non Linear Correlation},
    grid=both,
    width=8cm,
    height=6cm,
    xtick=\empty,
    ytick=\empty,
]
% Scatter Points
\addplot[
    only marks,
    mark=*,
] table[
    col sep=comma,
    x=X,
    y=Y
] {non_linear_correlation.csv};
% Regression Line
\addplot[
    red,
    thick,
    opacity=0.4
] table[
    col sep=comma,
    x=X,
    y=Y,
    y={create col/linear regression={y=Y}}
] {non_linear_correlation.csv};
\end{axis}
\end{tikzpicture}
\end{center}

When we analyze this scatter plot, we can see:
\begin{itemize}
       \item While data is following a pattern (in this case exponential), it is not a straight line
\end{itemize}
\vfill
\begin{center}
       (Continues on next page.)
\end{center}

%%%%%%%%%% Page 5 %%%%%%%%%%
\newpage
\subsection{Weak or No Linear Correlation}
%% Weak or No Linear Correlation - Scatter Plot
% Graph Outline
\begin{center}
\begin{tikzpicture}
\begin{axis}[
    xlabel={$x$},
    ylabel={$y$},
    title={Weak or No Linear Correlation},
    grid=both,
    width=8cm,
    height=6cm,
    xtick=\empty,
    ytick=\empty,
]
% Scatter Points
\addplot[
    only marks,
    mark=*,
] table[
    col sep=comma,
    x=X,
    y=Y
] {weak_linear_correlation.csv};
% Regression Line
\addplot[
    red,
    thick,
    opacity=0.4
] table[
    col sep=comma,
    x=X,
    y=Y,
    y={create col/linear regression={y=Y}}
] {weak_linear_correlation.csv};
\end{axis}
\end{tikzpicture}
\end{center}

When we analyze this scatter plot, we can see:
\begin{itemize}
       \item The data has no clear pattern (a diffuse cloud)
\end{itemize}




\end{document}               